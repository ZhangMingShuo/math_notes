\documentclass[a4paper,fleqn]{article}
\usepackage{amsmath}
\usepackage{geometry}
\usepackage{type1cm}
\usepackage[UTF8]{ctex}

\geometry{a4paper,left=1cm,right=1cm,top=1cm,bottom=1cm}
%\title{La}

\begin{document}
%  \vspace*{-105pt}
%  {\let\newpage\relax\maketitle}

  \fontsize{9pt}{13.5pt}\selectfont
%输入数学公式
\(
	(u+v)^{(n)}=u^{(n)}+v^{(n)}
\qquad 
	(uv)^{(n)}=C_n^0u^{(n)}+C_n^1u^{(n-1)}v^\prime+ \cdots + C_n^nuv^{(n)}  
\qquad
	(sinx)^{(n)}=sin(x+\frac{n\pi}{2})  
\qquad
	(cosx)^{(n)}=coss(x+\frac{n\pi}{2})  
\)
\begin{displaymath} 
	\frac{1}{(ax+b)^{(n)}}= \frac{(-1)^nn!a^n}{(ax+b)^{(n+1)}} 
\end{displaymath} \\
设y=f(x)可导且\(f^\prime(x)\neq 0, x= \varphi(y)\)为反函数,则\(x=\varphi(y)\)可导,且
\(
\varphi^\prime(y)=\frac{1}{f^\prime(x)}
\) \\
设y=f(x)二阶可导且\(f^\prime(x)\neq 0, x= \varphi(y)\)为反函数,则\(x=\varphi(y)\)二阶可导,且
\(
\varphi^{\prime\prime}(y)=-\frac{f^{\prime\prime}(x)}{f^{\prime3}(x)}
\) \\
\(x \to 0 \)常用的等价无穷小 
\(
x \sim sinx \sim tanx \sim arcsinx \sim arctanx \sim ln(1+x) \sim e^x-1 ,\qquad 1-cosx \sim \frac{x^2}{2},1-cos^ax \sim \frac{a}{2} x^2
\)\\
\(
(1+x)^a-1 \sim ax ,\qquad 
a^x-1 \sim xlna
\) \\
\( x\to 0 \)常用的麦克劳林公式
\( 
e^x=1+x+\frac{x^2}{2!}+\frac{x^3}{3!}+\dots+\frac{x^n}{n!}+o(x^n) \qquad 
sinx=x-\frac{x^3}{3!}+\frac{x^5}{5!}-\frac{x^7}{7!}+\dots+\frac{(-1)^n}{(2n+1)!}x^{2n+1}+o(x^{2n+1})
\)\\
\(
cosx=1-\frac{x^2}{2!}+\frac{x^4}{4!}-\frac{x^6}{6!}+\dots+\frac{(-1)^n}{(2n)!}x^{2n}+o(x^{2n}) \qquad 
\frac{1}{1-x}=1+x+x^2+x^3+x^4+\dots+x^n+o(x^n) \)\\
\(
\frac{1}{1+x}=1-x+x^2-x^3+x^4+\dots+\frac{(-1)^{n-1}}{n}x^n+o(x^n)
\)\\
\(
ln(1+x)=x-\frac{x^2}{2}+\frac{x^3}{3}-\dots+\frac{(-1)^{n-1}}{n}x^n+o(x^n) \qquad
(1+x)^a=1+ax+\frac{a(a-1)}{2!}x^2+\frac{a(a-1)(a-2)}{3!}x^3+\dots+\frac{a(a-1)\dots(a+1-n)}{n!}x^n
\)
\end{document}
