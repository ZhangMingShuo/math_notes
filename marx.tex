
\documentclass[a4paper,fleqn]{article}
\usepackage{geometry}
\usepackage[UTF8]{ctex}
\usepackage{type1cm}
\geometry{a4paper,left=0.5cm,right=0.5cm,top=0.5cm,bottom=0.5cm}
\begin{document}
%\fontsize{9pt}{13.5pt}\selectfont
1、马克思主义包括:1)马克思主义哲学 2)马克思主义政治经济学 3)科学社会主义 \\
2、资本主义经济的发展,为马克思主义的产生提供了经济、社会历史条件。 \\
3、马克思恩格斯批判吸收了德国古典哲学、英国古典政治经济学和法国、英国空想社会主义的合理成分,创立了唯物史观和剩余价值学说,把社会主义由空想变成科学。 \\
4、习近平新时代中国特色社会主义思想,1)是马克思主义中国化最新理论成果;2)是 21世纪的马克思主义;3)当代中国的马克思主义。 \\
5、马克思主义具有鲜明的科学性、革命性、实践性、人民性和发展开放性。 \\
6、科学性是指马克思主义是对自然、社会和人类思维发展本质和规律的正确反映。 \\
7、革命性是指马克思主义的集中表现了彻底的批判精神和鲜明的无产阶级立场。 \\
8、实践性是指马克思主义是从实践中来,到实践中去,在实践中接受检验,并随实践而不断发展的学说。实践性是马克思主义区别于其他理论的显著特征。 \\
9、人民性是指人民至上是马克思主义的政治立场。马克思主义政党把人民放在心中最高位置,一切奋斗都致力于实现最广大人民的根本利益。 \\
10、发展性是马克思主义是不断发展的学说,具有与时俱进的理论品质。 \\
11、开放性是指马克思主义不断吸收人类历史上一切优秀思想文化成果而不断丰富自己。 \\
12、习近平指出,1)马克思主义是科学的理论,创造性的揭示了人类社会发展规律。2)马克思主义是人民的理论,第一次创立了人们实现自身解放的思想体系。3)马克思主义是实践的理论,指引着人们改造世界的行动。4)马克思主义是不断发展的开放的理论,始终站在时代前沿。 \\
13、马克思主义的当代价值:1)马克思主义是我们观察当代世界变化认识工具。2)马克思是指引当代中国发展的行动指南。3)马克思主义是引领人类社会进步的科学真理。 \\
14、新时代仍然要学习和实践马克思主义,习近平指出,马克思主义始终是我们党和国家的指导思想,是我们认识世界、把握规律、追求真理、改造世界的强大思想武器。 \\
15、恩格斯明确提出思维和存在的关系问题是哲学的基本问题。哲学基本问题包括思维和存在何者第一性与思维和存在有无同一性。 \\
16、根据对哲学基本问题第一方面的不同回答,哲学可划分为唯物主义和唯心主义两派。唯物主义把世界的本源归结为物质主张物质第一性、意识第二性,意识是物质的产物。唯心主义把世界的本源归结为精神主张意识第一性、物质第二性,物质是意识的产物。 \\
17、唯物主义包括:古代朴素唯物主义(金木水火土气)近代形而上学唯物主义(原子等)现代辩证唯物主义(联系、发展、全面的观点)唯心主义包括客观唯心和主观唯心。 \\
18、根据对哲学基本问题第二方面的不同回答,这些可划分为可知论和不可知论。可知论认为世界是可以被认识的,思维和存在具有同一性。不可知论认为世界是不能被人所认识或不能被完全认识的,否认思维和存在的同一性。 \\
19、辩证法:坚持用联系的、发展的观点看世界,认为发展的根本原因在于事物的内部矛盾。 \\
20、形而上学:用孤立的、静止的观点看问题,否认事物内部矛盾的存在和作用。 \\
21、物质范畴是唯物主义哲学关于世界本原和统一性的最高抽象,是唯物主义世界观的基石。 \\
22、列宁指出,“物质是标志客观实在的哲学范畴,这种客观实在是人通过感觉感知的,它不依赖于我们的感觉而存在,为我们的感觉所复写、摄影、反映。”列宁是从物质和意识的关系上给物质下定义的。物质的唯一特性是客观实在性。 \\
23、物质的根本属性是运动。 \\
24、运动和物质不可分割,设想不运动的物质导致形上学;设想无物质的运动将导致唯心。 \\
25、运动是绝对的,静止是相对的,静止是运动的衡量尺度。运动的绝对性体现了物质运动的无条件性,静止的相对性体现了物质运动的有条件性。运动和静止相互依赖、相互渗透、相互包涵,“动中有静,静中有动”。 \\
26、时间和空间是物质运动的存在形式。时间是物质运动过程的持续性、顺序性,特点是一维性。空间是物质运动的伸长性、广延性,特点是三维性。 \\
27、时空的特点:1)具体物质形态的时空是有限的 2)而整个物质世界的时空是无限的 3)物质运动时间和空间的客观实在性是绝对的 4)物质运动时间和空间的具体特性是相对的。 \\
28、实践是使物质世界分化为自然界与人类社会的历史前提,又是使自然界与人类社会统一起来的现实基础。 \\
29、社会生活的实践性主要表现在:1)实践是社会关系形成的基础。2)实践形成了社会生活的基本领域。 3)实践构成了社会发展的动力。 \\
30、人与自然的关系越来越重要,生态环境已日益成为人类反思自身活动的重要前提。 \\
31、生态、环境、人口、资源等全球危机问题,实际上人与自然关系的严重失衡。 \\
32、人与自然是生命共同体,人类必须尊重自然、顺应自然、保护自然。人类只有遵循自然规律才能有效防止在开发利用自然上走弯路,人类对大自然的伤害最终会伤及人类自身,这是无法抗拒的规律。 \\
33、物质与意识的辩证关系:物质决定意识,意识依赖于物质并反作用于物质。 \\
34、意识的起源:意识是自然界长期发展的产物,也是社会历史的产物。社会实践特别是劳动在意识的产生和发展中起着决定性的作用。 \\
35、劳动为意识的产生提供了客观的需要,语言促进了意识的发展,语言是意识的物质外壳。 \\
36、意识的本质:意识是人脑的机能和属性,是客观世界的主观印象。意识在内容上是客观的,在形式上是主观的,是客观内容和主观形式的统一。意识是物质的产物,但不是物质本身。 \\
37、意识对物质具有反作用:意识的能动作用,是人特有的积极的认识世界和改造世界的能力和活动。主要表现在:1)意识活动具有目的性计划性。2)意识活动具有创造性。3)意识具有指导实践改造客观世界的作用。4)意识具有控制人的行为和生理活动的作用。 \\
38、必须处理好主观能动性和客观规律的关系:1)尊重客观规律是发挥主观能动性的前提。2)只有充分发挥主观能动性,才能正确认识和利用客观规律。3)实践是客观规律性与主观能动性统一的基础。 \\
39、人们要正确发挥主观能动性,应当注意以下几点:1)从实际出发,努力认识和把握事物的发展规律。 2)实践是发挥人的主观能动作用的基本途径。3)主观能动作用的发挥,还依赖于一定的物质条件和物质手段。 \\
40、世界的统一性在于它的物质性。1)世界是统一的,即世界的本原是一个。2)世界的统一性在于它的物质性。3)物质世界的统一性是多样性的统一,不是单一的无差别的统一。 \\
41、世界的物质统一性还体现在人类社会,也统一于物质。社会的物质性表现在:1)人类社会依赖于自然界,是整个物质世界的组成部分。2)人类谋取物质生活资料的实践活动虽然有意识作指导,但仍然是物质性的活动。3)物质资料的生产方式是人类社会存在和发展的基础,集中体现着人类社会的物质性。 \\
42、联系和发展的观点是唯物辩证法的总观点和总特征。 \\
43、联系是指事物内部各要素之间和事物之间的相互影响、相互制约、相互作用的关系。联系的特点有:1) 联系具有客观性、2)普遍性、3)多样性、4)条件性。 \\
44、发展是前进上升的运动,发展的实质是新事物的产生和旧事物的灭亡。新事物是指合乎历史前进方向、 具有远大前途的东西。旧事物是指丧失历史必然性、日趋灭亡的东西。 \\
45、新事物必然战胜旧事物:1)新事物具有新的结构和功能,能适应已经变化了的环境和条件。2)新事物对旧事物的扬弃,并添加了旧事物所不能容纳的新内容。3)在社会历史领域内,新生事物符合广大人民群众的根本利益和要求。 \\
46、事物发展的过程,1)形式上看,是事物在时间上的持续性和空间上的广延性的交替。2)内容上看,是事物在运动形式、形态、结构、功能和关系上的更新。 \\
47、对立统一规律是事物发展的根本规律,矛盾规律揭示了事物变化发展的动力和源泉,是唯物辩证法的实质和核心。 \\
48、对立统一规律又称矛盾规律,矛盾是辩证法的核心概念。矛盾是反映事物内部和事物之间对立统一关系的哲学范畴。 \\
49、同一性和斗争性是矛盾的两种基本属性。1)同一性是指矛盾双方相互依存、相互贯通的性质和趋势。 2)斗争性是指矛盾双方相互排斥、相互分离的性质和趋势。 \\
50、同一性和斗争性是相互联结,相互制约的。1)同性不能脱离斗争性而存在,没有斗争性就没有同一性。 因为同一性是以差别和对立为前提的,是包含着差别和对立的具体的同一。2)斗争性也不能脱离同一性而存在,斗争性寓于同一性之中。3)矛盾的斗争性是无条件的绝对的,同一性是有条件的相对的。 \\
51、同一性对于事物发展的作用主要是:(和谐、稳定、协调、双赢、共赢)1)矛盾双方可以利用对方的发展使自己获得发展。2)矛盾双方可以相互吸取有利于自身的因素而得到发展。3)矛盾双方向着自己的对立面转化而得到发展,并规定着事物发展的方向。 \\
52、矛盾的普遍性:矛盾存在于一切事物中,存在一切事物发展过程始终,即矛盾无处不在,无时不有。承认矛盾的普遍性是一切科学认识的首要前提。(矛盾分析法是唯物辩证法的根本方法。) \\
54、矛盾特殊性:具体事物在其运动中的矛盾及每一矛盾的各个方面都有其特点。表现为:1)不同事物的矛盾各有其特点。2)同一事物的矛盾在不同发展过程和发展阶段各有其不同特点。3)构成事物的诸多矛盾以及每一矛盾的不同方面各有不同的性质地位和作用。 \\
55、矛盾解决的形式:1)矛盾一方克服另一方矛盾。2)双方同归于尽。3)矛盾双方形成协同运动的新形势。4)矛盾双方融合成一个新事物。 \\
56、矛盾的普遍性和特殊性即矛盾的共性和矛盾的个性是辩证统一的。1)矛盾的共性是无条件的、绝对的,矛盾的个性是有条件的、相对的。2)任何现实存在的事物都是共性和个性的有机统一,共性寓于个性之中。(不能说个性寓于共性之中。)3)矛盾的共性和个性、绝对和相对的道理是关于事物矛盾问题的精髓。 \\
57、质是一事物成为自身并区别于其他事物的内在规定性。量是事物的规模、程度、速度等可以用数量关系表示的规定性。度是保持事物质的稳定性的数量界限,即事物的范围、幅度和限度。 \\
58、质与事物具有直接同一性。 \\
59、认识度才能确切的把握事物的质,认识和处理问题时要掌握适度原则。 \\
60、量变和质变的辩证关系是:1)量变是质变的必要准备,没有量的积累质变就不会发生。2)质变是量变的必然结果,量变达到一定程度必然引起质变。(坚持就是胜利。)3)量变和质变是相互渗透的。一方面,量变过程当中有阶段性和局部性的部分质变;另一方面在质变的过程当中也有旧质在量上的收缩和新 质在量上的扩张。 \\
61、肯定因素是维持现存事物存在的因素,否定因素是促使现存事物灭亡的因素。 \\
62、辩证否定观的基本内容是:1)否定是事物的自我否定,是事物内部矛盾运动的结果。2)否定是事物发展的环节。只有经过否定旧事物才能向新事物转变。3)否定是新旧事物联系的环节。4)辩证否定观的实质是“扬弃”。 \\
63、否定之否定规律揭示了事物变化发展的方向和道路,是新事物战胜旧事物的哲学依据。三个阶段即“肯定 ——否定——否定之否定”。从内容上看,是自己发展自己,自己完善自己的过程。从形式上看是螺旋式上升或波浪式前进的过程。方向是前进上升的,道路是曲折迂回的。\\
64、客观辩证法是指客观事物或客观存在的辩证法,即客观事物以相互作用、相互联系的形式呈现出的各种物质形态的辩证运动和发展规律。 \\
65、主观辩证法是指人类认识和思维运动的辩证法,即以概念作为思维细胞的辩证思维运动和发展的规律。(不能说主观辩证法是唯心辩证法。) \\
66、归纳与演绎:是最初、最基本的思维方法。归纳是从个别上升到一般的方法,演绎是从一般到个别的方法。归纳是演绎的基础,演绎是归纳的前提。 \\
67、分析与综合:分析是在思维过程中把认识的对象分解为不同的组成部分、方面、特性等,对他们分别加以研究。综合则是把分解出来的不同的部分、方面按其客观的次序、结构组成一个整体,从而达到对事物整体的认识。(整体和部分/系统和要素) \\
68、抽象与具体,是辩证思维的高级形式。 \\
69、逻辑与历史:逻辑指的是理性思维和抽象思维,历史包含两层意思,一是指客观现实的历史发展过程,二是指人类认识的历史发展过程。逻辑是历史在理论上的再现,是“修正过”的历史。 \\
70、内容与形式是揭示事物内在要素同这些要素的结构和表现方式的关系范畴。内容是构成事物一切要素的总和,是事物存在的基础。形式是内容诸要素相互结合的结构和表现方式。 \\
71、内容和形式是相互依赖、不可分割的:1)任何事物的内容都有一定的形式,任何形式也都有一定的内容,没有无内容的空洞的形式,也没有无形式的纯粹的内容。(多选)2)事物的内容是无限丰富的,事物的形式也是多种多样的。 \\
72、内容和形式相互作用、相互影响。内容决定形式,形式反作用于内容。当形式适合内容时,对内容的发展会起积极的推动作用;当形式不适合内容时,对内容的发展起着消极的阻碍作用。 \\
73、现象与本质是揭示客观事物的内在联系和外部表现相互关系的范畴。本质和现象有区别:(现象和本质都是客观的,他们不是反映与被反映关系,而是表现与被表现的关系。)1)本质是事物的根本性质,是构成事物诸要素之间的内在联系。现象是事物的外部联系和表面特征,是事物的外在表现。现象可以区 分为真相和假象。2)本质是一般的、普通的,现象是个别的、具体的。3)本质是相对稳定的,现象是多变易逝的。4)本质深藏于事物的内部,只有通过理性思维才能把握,而现象是表面、外显的,可以直接为人的感官所感知。 \\
74、原因和结果是揭示事物的前后相继彼此制约的关系范畴。引起某种现象的现象叫原因,被某种现象所引 起的现象叫结果。 \\
75、原因和结果的关系是辩证的:1)原因和结果的区分既是确定的又是不确定的。2)原因和结果相互作用,原因产生结果,结果反过来影响原因,互为因果。3)原因和结果相互渗透,结果存在原因之中,原因表现在结果之中。4)原因和结果的关系是复杂多样的,有一因多果、同因一果、一果多因、异因同 果、多因多果、复合因果。 \\
76、必然和偶然是揭示客观事物发生、发展、灭亡不同趋势的范畴。必然是一定发生、确定不移的趋势。偶然是可以出现、可以不出现,不确定的趋势。 \\
77、必然和偶然的联系:相互依存和相互转化。1)必然存在于偶然之中,通过大量的偶然表现出来,并为自己开辟道路。2)偶然背后隐藏着必然受必然的支配,偶然是必然的表现形式和补充。3)必然和偶然在一定条件下可以相互转化。(“必然的东西是偶然的,偶然的东西是必然的。”) \\
78、现实与可能是揭示事物的过去,现在和将来的相互关系的范畴。区分:1)可能和不可能(有无现实根据)2)现实的可能和抽象的可能(现实根据是否充分)3)好的可能和坏的可能(对人是否有利) \\
79、现实和可能相互区别:1)可能不等于现实,现实已经不是可能。2)现实是当下的客观存在,标志着事物的当前情况;可能是事物潜在的趋势,标志着事物发展方向。 \\
80、现实和可能相互转化:1)现实蕴藏着未来的发展方向,会不断产生出新的可能。2)可能包含着发展成为现实的因素和依据,一旦主客观条件成熟,可能就会转化为现实。 \\
81、实践是人类能动的改造世界的社会性的物质活动。 \\
81、实践的要素包含:主体、客体、中介。 \\
82、实践的主体和客体相互作用的关系包括:实践关系、认识关系、价值关系,其中实践关系是最根本的关系。 \\
83、实践基本特征有:1)直接现实性 2)自觉能动性 3)社会历史性  \\
84、实践的基本类型有:1)物质生产实践。2)社会政治实践。3)科学文化实践。其中物质生产实践是人类最基本的实践活动,决定着社会的基本性质和面貌。 \\
85、实践在认识活动中的决定作用:1)实践是认识的来源。2)实践是认识发展的动力。3)实践是认识的目的。4)实践是检验认识真理性的唯一标准。实践的观点是马克思主义认识论的第一的和基本的观点。 \\
86、在认识的本质问题上存在着两条根本对立的认识路线:1)坚持从物到感觉和思维的唯物主义路线,唯物主义反映论,认为认识是主体对客体的反映。2)坚持从思维和感觉到物的唯心主义路线,唯心主义先验论,把认识看作是先于物质的东西。 \\
87、辩证唯物主义和旧唯物主义对认识本质的不同回答:1)旧唯物主义认识论以感性直观为基础,把人的认识看成是消极的、被动地反映外界对象。有两个严重缺陷:离开实践考察认识问题;离开辩证法来考察认识问题。2)辩证唯物主义认识论坚持以科学的社会实践为特征的实践观。 \\
88、认识的本质,认识是主体在实践基础上对客体的能动反应。 \\
89、认识的过程包含两次飞跃:1)第一次飞跃是在实践基础上认识活动由感性认识能动的飞跃到理性认识。 2)第二次飞跃,也是更为重要的一次飞跃,即从认识到实践。 \\
90、感性认识包括:感觉、知觉、表象三种形式。感性认识是认识的初级阶段,直接性是感性认识的突出特点。 \\
91、理性认识是认识的高级阶段,包括概念、判断、推理三种形式。理性认识的特点是它的间接性和抽象性。 \\
92、感性认识和理性认识的联系:(“熟知不等于真知”,“感觉到了的东西我们不能立刻理解它”)1)感性认识有待于发展和深化为理性认识。2)理性认识依赖于感性认识。3)感性认识和理性认识相互渗透、相互包含。 \\
93、感性认识和理性认识是辩证统一的,是在实践的基础上形成的。如果割裂二者辩证统一关系,就会走向唯理论或经验论。1)唯理论就会犯教条主义错误。2)经验论就会犯经验主义错误。 \\
94、理性因素是指人的理性直观、理性思维等能力。它在认识活动当中的作用主要有:1)指导作用。2)解释作用。3)预见作用。 \\
95、非理性因素主要是指认识主体的情感和意志。如联想、想象、猜测、顿悟、灵感等。非理性因素对人的认识能力和认识活动具有:1)激活作用。2)驱动作用。3)控制作用。 \\
96、真理是人们对于客观事物及其规律的正确反映。真理具有客观性,凡真理都是客观真理。真理的客观性决定了真理的一元性。真理的一元性是指对于特定认识客体来说,真理只有一个,它不因主体认识的差别和变化而改变。 \\
97、真理是个发展过程,既是绝对的又是相对的。真理的绝对性是指真理的内容表明了主客观统一的确定性和发展的无限性。真理的相对性是指在一定条件下对事物的客观过程及其发展规律的正确认识是有限度的。 \\
98、真理的绝对性和相对性是辩证统一的。1)二者相互依存。人们对于客观事物及其本质的规律的每一个正确认识,都是在一定范围内、一定程度上、一定条件下的认识,因而必然是相对的和有限性的;但是在这一定范围内、一定程度上、一定条件下,他又是对客观对象的正确反映,因而它又是无条件的、绝对的。 2)二者相互包含。一是真理的绝对性寓于真理的相对性之中。二是真理的相对性必然包含并表现着真理 的绝对性。所以绝对真理和相对真理是不可分的,没有离开绝对真理的相对真理,也没有离开相对真理的 绝对真理。真理是从真理的相对性走向绝对性、接近绝对性的过程。 \\
99、真理与谬误:真理和谬误是相伴而生的,人类在探索真理的过程中,难免发生谬误。真理和谬误决定于认识的内容是否如实的反映了客观事物,它们是性质对立的两种认识。(不是认识过程的两个阶段)真理和谬误又是统一的,他们相互依存、相互转换。1)真理和谬误相比较而言,没有真理就无所谓谬误。 没有谬误也无所谓真理。2)真理中包含着以后会暴露出来的错误因素,谬误中也隐藏着以后会显露出来 的真理的成分。3)真理和谬误在一定范围内的对立是绝对的,但超出一定范围,它们就会互相转换。 \\
100、实践是检验真理的唯一标准,这是有实践的本性(主观和客观一致)和实践的特点所决定的。 \\
101、实践标准的确定性与不确定性。1)实践标准的确定性即绝对性是指,实践是检验真理的唯一标准。2)实践标准的不确定性是指,社会实践总会受历史条件的限制。实践检验真理是一个过程,不是一次完成的。已被实践检验过的真理还要继续接受实践的检验。 \\
102、价值是指在实践基础上形成的主体和客体之间的意义关系,是客体对个人、群体乃至整个社会生活和活动所具有的积极意义。价值体现的是主体和客体之间的一种特定关系。价值离不开主体的需要,也离不开客体的特性。价值既具有主体性的特征,又具有客观基础。 \\
103、价值的基本特性:主体性、客观性、多维性、社会历史性。 \\
104、价值评价是主体对客体价值以及价值大小所作的评判或判断。其基本特点主要有:1(评价以主客体的价值关系为认识对象。2)评价结果与评价主体直接相关。3)评价结果的正确与否依赖于对客体状况和主体需要的认识。 \\
105、人们的实践活动总是受到真理尺度和价值尺度的制约。1)真理尺度是指在实践中人们必须遵循正确反映客观事物本质和规律的真理。2)价值尺度是指在实践中人们都是按照自己的尺度和需要去认识世界和改造世界。3)任何成功的实践都是真理尺度和价值尺度的统一。 \\
106、真理尺度与价值尺度是紧密联系、不可分割的辩证统一关系:1)价值尺度必须以真理尺度为前提。2)人类自身需要的内在尺度,推动着人们不断发现新的真理。 \\
107、认识世界和改造世界,是人类创造历史的两种基本活动。认识的任务不仅在于解释世界,更重要的在于为改造世界提供理论指导。 \\
108、认识世界的目的是为了改造世界,而改造世界又包括改造客观世界和主观世界。1)客观世界是指“物质的、可以感知的世界”,包含自然存在和社会存在两个部分。2)主观世界是指人的意识、观念世界,是人的头脑反映和把握物质世界的精神活动的总和。 \\
109、从必然走向自由:1)必然性即规律性,指不依赖于人的意识而存在的自然和社会发展所固有的客观规律。2)自由是对必然的认识和对客观世界的改造。3)认识必然,争取自由,是人类认识世界和改造世界的根本目标,是一个世界历史性的过程。4)由必然到自由表现为人类不断的从必然王国走向自由王国 的过程。自由是历史发展的产物。5)自由是有条件的:认识条件、实践条件。自由与必然的关系贯穿于 人类存在和发展的始终,也是人类存在和发展的永恒动力。 \\
110、社会存在与社会意识的关系问题是,社会历史观的基本问题。在对待社会历史发展及其规律问题上,存在两种对立观点,唯物史观和唯心史观。 \\
111、社会存在也称物质生活条件是社会生活的物质方面,主要包括:a、物质生产方式 b、自然地理环境 c、人口因素。 \\
112、社会意识是社会生活的精神方面,是社会存在的反映,社会意识分为个人意识和群体意识,社会心理和社会意识形式。 \\
113、意识形态是指反映社会经济关系,切切关系的社会意识,主要包括政治法律思想、道德、艺术、宗教、哲学等。 \\
114、社会存在和社会意识是辩证统一的。1)社会存在决定社会意识,社会意识是社会存在的反映,并反作用于社会存在。2)社会存在是社会意识内容的客观来源。3)社会意识是人们进行社会物质交往的产物。 4)随着社会存在的发展,社会意识也相应的或迟或早地发生变化和发展。 \\
115、社会意识具有相对独立性:1)社会意识与社会存在发展的不完全同步性和不平衡性。2)社会意识内部各种形式之间的相互影响及各自具有的历史继承性。3)社会意识对社会存在的能动的反作用。这是社会意识相对独立性的突出表现。 \\
116、物质生产方式是社会发展的决定力量。是劳动者和劳动资料结合的特殊方式,是生产力和生产关系的统一。 \\
117、生产力是人类在生产实践中形成的,改造和影响自然以使其适应社会需要的物质力量。生产力具有客观现实性和社会历史性。生产力的性质是生产力的质的规定性,它取决于生产的物质技术性质,主要是劳动资料的性质。 \\
118、生产力的基本要素:1)包括劳动资料(最重要的是生产工具,它是区分社会经济时代的客观依据)2)劳动对象(劳动对象和劳动资料合称生产资料)3)劳动者(劳动者是生产力中最活跃的因素。) \\
119、科学技术日益成为生产发展的决定性因素,科学技术是先进生产力的集中体现和主要标志,是第一生产力。 \\
120、生产关系是人们在物质生产过程中形成的不以人的意志为转移的经济关系,是社会关系中最基本的关系。包括生产资料所有制关系、生产中人与人的关系和产品分配关系。在生产关系中,生产资料所有制是最基本的。 \\
121、生产关系是一种客观的物质的社会关系。(占统治地位的生产关系的总和是经济基础,是社会骨骼系统) \\
122、生产关系具有客观性,本质上是生产力的社会存在形式,人们不能任意改变生产关系。生产关系是一种物质利益关系,它体现着人们之间的物质经济利益。 \\
123、生产力决定生产关系:1)生产力状况决定生产关系的性质。2)生产力的发展决定生产关系的变革。 \\
124、生产关系反作用于生产力:1)当适合时会起到推动作用。2)不适合时会阻碍生产力的发展。3)当不变更生产关系生产力就不能继续发展时,生产关系对生产力的反作用尤为突出。 \\
125、判断一种生产关系是否优越的标准在于,这种生产关系对生产力是适合的还是不适合的,是促进的还是阻碍生产力发展。 \\
126、生产关系一定要适合生产力状况的规律,是人类社会发展的基本规律。 \\
127、经济基础是由社会一定发展阶段的生产力所决定的生产关系的总和。 \\
128、上层建筑是指在一定经济基础之上的意识形态以及相应的制度、组织和设施。1)观念上层建筑,包括政治法律思想、道德、艺术、宗教、哲学等。2)政治上层建筑,包括国家政治制度、立法司法制度和行政制度、国家政权机构、政党、军队、警察、法庭、监狱等政治组织形态和设施。 \\
129、政治上层建筑是在一定意识形态指导下建立起来的,是统治阶级意志的体现。在整个上层建筑中政治上层建筑居主导地位,国家政权是它的核心。 \\
130、国家不是从来就有的,它是社会发展到一定历史阶段的产物,是阶级矛盾不可调和的产物。国家的实质是一个阶级统治另一个阶级的工具。 \\
131、经济基础决定上层建筑具体表现在:1)经济基础的需要决定上层建筑的产生。2)经济基础的性质决定上层建筑的性质。3)经济基础的变化发展决定上层建筑的变化发展及其方向。 \\
132、上层建筑对经济基础具有反作用。这种反作用集中表现在为自己的经济基础服务。上层建筑反作用的性质,取决于它所服务的经济基础的性质,取决于它是否有利于生产力的发展。 \\
133、社会形态是同生产力发展一定阶段相适应的经济基础和上层建筑的统一体。社会形态包括社会的经济形态、政治形态和意识形态。 \\
134、社会形态更替的统一性和多样性:1)统一性是社会形态运动由低级到高级、由简单到复杂的过程,表现为社会形态依次更替。2)多样性是指不同的民族可以超越一种或几种社会形态而跳跃的向前发展。 (例如中国) \\
135、意识形态更替的必然性和人们的历史选择性:1)社会形态更替的客观必然性是指社会依次更替的过程和规律是客观的,发展的基本趋势是确定不移的。社会更替归根到底是社会基本矛盾运动的结果。2)人们的历史选择性包含 3层含义:a、社会发展客观必然性为人们的历史选择性提供了基础、范围和可能性 空间。b、社会形态更替的过程也是一个合目的性与合规律性相统一的过程。c、人们的历史选择性归根到 底是人民群众的选择性。最终取决于人民群众的根本利益、根本意愿以及社会发展规律的把握顺应程 度。 \\
136、社会形态还表现为历史的前进性和曲折性、顺序性与跨越性的统一。 \\
137、社会基本矛盾是社会发展的根本动力。生产力和生产关系、经济基础和上层建筑的矛盾,是社会基本矛盾。生产力是社会基本矛盾运动中最基本的动力因素,是人类社会发展和进步的最终决定力量,生产力是社会发展的根本内容。 \\
138、阶级是社会发展到一定阶段及生产有所发展而又发展不足的产物。阶级的本质是与特定的生产关系相联系的,处于不同地位和社会集团或人群共同体。 \\
139、阶级对立的实质,是社会上一部分人拥有生产资料因而占有另一部分人的劳动。 \\
140、阶级斗争是社会基本矛盾在阶级社会中的直接表现,是阶级社会发展的直接动力。阶级斗争对阶级社会发展的推动作用突出的表现为在社会形态的更替中。 \\
141、社会革命既是社会基本矛盾运动的结果,又是推动社会发展特别是社会形态更替的重要动力。社会革命的实质是,革命阶级推翻反动阶级的统治,用新的社会制度代替旧的社会制度,解放生产力,推动社会发展。 \\
142、改革是同一种社会形态发展过程中的量变和部分质变,是推进社会发展的又一重要动力。中国的社会主义改革是一场广泛深刻的伟大变革,是社会主义制度自我完善和自我发展。从对我国社会生活的深远影响而言,则可以说是一场伟大的革命。 \\
143、科学技术革命是社会动力体系中的一种重要动力,是“历史的有力杠杆”。 \\
144、历史的创造者是人民群众还是英雄?这是唯物史观与唯心史观的分水岭。 \\
145、唯物史观立足于现实的人及其本质来把握历史创造者问题。现实的人是基于自身需要和社会需要而从事一定实践活动、处于一定社会关系中的、具有能动性的人。人的本质属性是社会属性。 \\
146、人民群众从质上说是指一切对社会历史发展起推动作用的人,从量上说是社会人口中的绝大多数。人民群众最稳定的主体部分始终是从事物质资料生产的劳动群众。在中国,凡是拥护、参与和推动中国特色社会主义事业的人都属于人民群众的范畴。 \\
147、在社会历史发展过程中,人民群众起着决定性的作用。人民群众是历史的主体,是历史的创造者。表现在:1)人民群众是物质财富的创造者。2)是社会精神财富的创造者。3)是社会变革的决定力量。 \\
148、历史是人民群众创造的,但人民群众创造历史的活动受到一定社会历史条件的制约。 \\
149、坚持以人民为中心的思想,创造性的运用了唯物史观关于人民群众创造历史的基本原理。 \\
150、商品是用来交换的,能满足人们某种需要的劳动产品,具有使用价值和价值两个因素。 \\
151、使用价值是商品的有用性,体现商品的自然属性。价值是凝结在商品中无差别的一般人类劳动体现商品生产者之间的社会关系。 \\
152、价值是交换价值的基础,交换价值是价值的表现形式。 \\
153、劳动的二重性:1)具体劳动形成使用价值。2)抽象劳动形成商品的价值实体。 \\
154、商品价值量,由生产商品所耗费的劳动量决定。与劳动时间成正比,与劳动生产率成反比。社会必要劳动时间决定商品的价值量。 \\
155、影响劳动生产率的因素包括:1)劳动者的平均熟练程度。2)科学技术的发展程度以及在生产中的应用。3)生产过程的社会结合。4)生产资料的规模和效能以及自然条件。 \\
156、商品的价值形式经历了 4个阶段:1)简单的偶然的价值形式。2)总和的或扩大的价值形式。3)一般的价值形式。4)货币形式。 \\
157、商品交换是以货币为媒介的,货币是在长期交换过程中形成的固定地充当一般等价物的商品。货币有 5 种职能:1)价值尺度。2)流通手段。3)贮藏手段。4)支付手段。5)世界货币。 \\
158、价值规律是商品经济的基本经济规律,它的内容和客观要求是:1)商品的价值量是由生产商品的社会必要劳动时间所决定的。2)商品的交换以价值量为基础,按照等价交换的原则进行。 \\
159、价值规律的表现形式是商品的价格围绕价值自发波动。   \\
160、价值规律在商品经济中的作用主要有三个方面:1)自发地调节生产资料和劳动力在社会各生产部门之间的分配比例。2)自发地刺激社会生产力发展。3)自发的调节社会收入分配。 \\
161、价值规律在对经济活动进行自发调节时,会产生一些消极后果:1)导致社会资源浪费。2)导致收入两极分化。3)阻碍技术的进步。 \\
162、只有商品变为货币私人劳动才能转化为社会劳动。私人劳动和社会劳动的矛盾构成私有制商品经济的基本矛盾。 \\
163、私有制商品经济的基本矛盾进一步发展成资本主义的基本矛盾即生产资料的资本主义私人占有和生产社会化之间的矛盾,这一矛盾的不断运动使得资本主义最终被社会主义代替。 \\
164、资本原始积累,主要通过两个途径进行:1)用暴力手段剥夺农民土地。2)用暴力手段掠夺货币财富。 \\
165、劳动力成为商品的必备,两个重要条件:第一,劳动者是自由人,第二,劳动者一无所有。  \\
166、劳动力成为商品是货币转化为资本的前提。 \\
167、劳动力商品的价值包含三个部分:1)维持劳动者本人生存所需的生活资料的价值。2)维持劳动者家属的生存所必需的生活资料的价值。3)劳动者接受教育和训练所支出的费用。 \\
168、劳动力的价值构成包含历史和道德的因素,劳动力价值的最低界限是由不可缺少的生活资料的价值决 定。 \\
169、劳动力商品的使用价值就是劳动,它是价值的源泉,并且是大于自身价值的价值源泉。 \\
170、货币所有者购买到劳动力这种特殊商品,能够获得剩余价值货币,也就转化为资本。在资本主义条件下,资本家购买的是雇佣工人的劳动力,而不是劳动。(劳动力成为商品,才能用劳动力生产剩余价值。) \\
171、资本家凭借对生产资料的占有,在等价交换原则的掩盖下,雇佣工人从事劳动,占有雇佣工人的剩余价值,这就是资本主义所有制的实质。 \\
172、生产过程的两重性:资本主义生产以获取剩余价值为目的。其生产过程是生产使用价值的劳动过程和生产剩余价值的价值增殖过程的统一。 \\
173、雇佣工人的劳动力分为必要劳动和剩余劳动。前者创造的价值,资本家以工资形式支付劳动力价值,后者创造的剩余价值,被资本家无偿占有。 \\
174、资本是可以带来剩余价值的价值。资本的本质不是物,而是一定历史社会形态下的生产关系。 \\
175、根据资本在剩余价值生产中所起的作用不同。可以区分为不变资本 c和可变资本 v。 \\
176、不变资本:以生产资料形式存在的资本,在生产过程中转移自己的物质形态,不发生价值增殖。可变资本:购买劳动力的那部分资本,在生产中不仅创造出劳动力价值,而且还实现了价值增殖。 \\
177、剩余价值率是剩余价值和可变资本的比率:m‘=m/v  \\
178、绝对剩余价值是必要劳动时间不变的条件下,延长工作日的长度而生产的剩余价值。(加班加点) \\
179、相对剩余价值是工作日长度不变的条件下,缩短必要劳动时间而延长剩余劳动时间所生产的剩余价值。 相对剩余价值生产以整个社会劳动生产率的提高为条件。 \\
180、超额剩余价值是企业由于提高劳动生产率而使商品个别价值低于社会价值的差额。 \\
181、资本积累:把剩余价值转化为资本,或者说剩余价值的资本化叫做资本积累。 \\
182、资本积累的本质,就是资本家不断利用无偿占有的工人创造的剩余价值,来扩大自己的这种规模,进一步扩大和加强对工人的剥削和统治。 \\
183、资本积累的源泉是剩余价值,影响资本积累的因素有:1)对工人的剥削程度。(高,多)2)劳动生产率水平的高低。(高,多)3)所用资本和所费资本的差额。(大,多)4)资本家垫付资本的大小。 (大,多) \\
184、资本的技术构成:是由生产的技术水平所决定的生产资料和劳动力之间的量的比例。(数量)资本的价值构成:从价值形式上看,资本构成的不变资本和可变资本之间的比例。(钱数)3)资本的有机构成由资本技术构成决定并反映资本技术构成变化的资本的价值构成。 \\
185、相对过剩人口有三种形式:1)流动的过剩人口。2)潜在的过剩人口。3)停滞的过剩人口。 \\
186、产业资本循环是指产业资本依次得经过三个阶段,采取三种职能形式,执行三种职能,实现了价值的增殖。产业资本运动依次经过购买、生产、售卖三个阶段。相应的采取货币资本、生产资本、商品资本三种职能形式。 \\
187、资本周转,资本是在运动中增殖的,资本周而复始,不断反复的循环。 \\
188、影响资本周转快慢的关键有两个因素:1)资本周转时间。2)生产资本中固定资本和流动资本的构成。要加快资本周转速度,获得更多剩余价值,就要缩短资本周转时间,加快流动资本周转速度。 \\
189、社会再生产的核心问题是社会总产品的实现问题,即社会总产品的价值补偿和实物补偿问题。 \\
190、社会总产品又叫社会总价值,包括转移价值 c、必要劳动价值 v和剩余价值 m三部分。 \\
191、社会再生产顺利进行,要求生产中所耗费的资本在价值上得到补偿,同时要求实际生产过程中所耗费的生产资料和消费资料得到实物的替换。 \\
192、工人工资是劳动力的价值或价格,这是资本主义工资的本质。资本家购买工人的劳动力是以货币工资形式支付的,工资表现为“劳动的价格”。 \\
193、利润是剩余价值的转化形式。利润率是剩余价值与全部预付资本的比率。p‘=m/c+v \\
194、影响利润率因素:1)剩余价值率。2)资本有机构成。3)资本周转速度。4)不变资本的节省。5)原材料价格变动。 \\
195、平均利润率是社会剩余价值总量同社会预付总资本的比率,通过部门之间的竞争形成。 \\
196、生产价格即商品的生产成本加平均利润。 \\
197、生产相对过剩是资本主义经济危机的本质特征。相对过剩是指相对于劳动人民有支付能力的需求来说社会生产的商品显得过剩,而不是与劳动人民的实际需求相比的绝对过剩。 \\
198、资本主义经济危机爆发的根本原因是资本主义的基本矛盾资本主义经济危机具有周期性,这是由资本主义基本矛盾运动的阶段性决定的。周期性特点一般包括 4个阶段:1)危机 2)萧条 3)复苏 4)高涨 \\
199、生产集中是指生产资料、劳动力和商品的生产日益集中于少数大企业的过程,其结果是大企业在社会生产中所占的比重不断增加。 \\
200、资本集中是指大资本吞并小资本,或有许多小资本合并而成大资本的过程,其结果是越来越多的资本为少数大资本所支配。 \\
201、垄断是指少数大企业为了获得高额利润,对一个或几个部门商品的生产、销售和价格实行操纵与控制。\\
202、最简单的、初级的垄断组织形式是短期价格协定。垄断组织的本质都是通过操纵垄断价格攫取高额垄断利润。 \\
203、垄断是从自由竞争中形成的,但是垄断并不能消除竞争。 \\
204、金融资本是由工业垄断资本和银行垄断资本融合在一起而形成的垄断资本。途径包括:1)金融联系。 2)资本参与。3)人事参与。 \\
205、金融寡头是指操纵国民经济命脉,并控制国家政权的少数垄断资本家或垄断资本家集团。金融寡头在经济中的统治,主要通过“参与制”实现;政治上通过“个人联合”来实现。金融寡头还通过建立政策咨询机构,掌握新闻科教文化等来左右国家的内政外交及社会生活。 \\
206、垄断价格=成本价格+平均利润+垄断利润。 \\
207、垄断价格的出现,使一些商品的价格经常高于或低于商品的价值或生产价格。但这并不否定价值规律。(垄断价格包括垄断高价和垄断低价。) \\
208、国家垄断资本主义是国家政权和私人垄断资本融合在一起的垄断资本主义。国家垄断资本主义的主要形式有:1)国家所有并直接经营的企业。2)国家与私人共有、合营企业。3)国家通过多种形式参与私人垄断资本的再生产过程。4)宏观调节。5)微观规制。微观规制有三种类型:a、反托拉斯法。b、公共 事业规制。c、社会经济规制。 \\
209、法人资本所有制有两种形式:1)企业法人资本所有制。2)机构法人资本所有制。 \\
210、随着社会生产力的发展和工人阶级反抗力量的不断壮大,资本家及其代理人开始采取一些缓和劳资关系的激励制度,使工人自觉地服从于资本家的意志。这些制度有:1)职工参与决策。2)终身雇佣。3)职工持股。\\
\end{document}